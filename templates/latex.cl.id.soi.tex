\documentclass[12pt]{cl.doc}
\usepackage{amsmath} % for \text
\usepackage{longtable} % Required for the longtable environment
\usepackage{float}
\usepackage{tabularray}
\usepackage{nameref} % Required package for using \nameref
\usepackage{tcolorbox}
\usepackage{background}
\usepackage{hyperref}
\title{\small JUDUL SOI}
\revisiondate{10 Februari 2025}
\revision{0.0}
\soinumber{CL-SOI-xxx-xx-000000-0}

\renewcommand{\theLogoValue}{cl.png}
\renewcommand{\figurename}{Gambar} % Change "Figure" to "Gambar"

% ---------------------------------------------------------------------------
% CUSTOM HYPHENATION
% ---------------------------------------------------------------------------
% Silahkan tambahkan custom pemenggalan kata  di dalam \hyphenation di bawah
% ini bila menemukan ada pemenggalan kata pada generated .pdf file yang salah
% di SOP/I ini.
% ---------------------------------------------------------------------------
\hyphenation{
    pe-ngen-da-li
    pe-mang-ka-san
}
% ---------------------------------------------------------------------------

\begin{document}
    \fontsize{10pt}{10pt}\selectfont
    \SetBgContents{}
    % Set up the background text
    \backgroundsetup{
      scale=7,              % Scale the text
      color=red,            % Color of the text
      opacity=0.15,         % Opacity of the text
      position={1.2, -1.5}, % Positioning
      contents=DRAFT        % The text to display
    }
    % ========================================
    % Saat sudah siap untuk proses pengesahan:
    % ----------------------------------------
    % 1. Comment, dengan menyisipkan % pada block \backgroundsetup{ s/d } di
    %    atas. Dengan editor VIM, cara-nya sebagai berikut:
    %     a) Saat di mode NORMAL, arahkan kursor ke posisi \ pada
    %        \backgroundsetup{ di atas.
    %     b) Tekan CTRL+v, lalu tekan j beberapa kali hingga kursor sampai pada
    %        posisi } yang sejajar dengan \ di bawah-nya.
    %     c) Shift+i, ketik %, dan lanjutkan dengan tekan Escape.
    % 2. Hapus % block \backgroundsetup{ s/d } di bawah. Dengan editor VIM,
    %    cara-nya sebagai berikut:
    %     a) Saat di mode NORMAL, arahkan kursor ke ke posisi \ pada
    %        \backgroundsetup{ di bawah.
    %     b) Tekan CTRL+v, lalu tekan j beberapa kali hingga kursor sampai pada
    %        posisi } yang sejajar dengan \ di bawah-nya.
    %     c) Tekan d.
    %
    %\backgroundsetup{
    %  scale=5,              % Scale the text
    %  color=red,            % Color of the text
    %  opacity=0.15,         % Opacity of the text
    %  position={1.7, -2}, % Positioning
    %  contents=CONFIDENTIAL % The text to display
    %}

    \begin{table}[H]
        \centering
        \small % make the font smaller
        \begin{tblr}{
                cell{1}{2-6}   = {c},
                cell{1}{3}     = {c=4}{},       % Baris 1 kolom 3 = Span 4 kolom lebar flexibel.
                cell{1}{5}     = {c=2}{},       % Baris 1 kolom 5 = Span 2 kolom lebar flexibel.
                cell{1}{1}     = {r=2}{1.75cm}, % Baris 1 kolom 1 = Span 2 baris, lebar 1.5cm.
                cell{4-5}{2-6} = {2.55cm},      % Pengaturan lebar kolom 2-6 pada baris 4-5 = 2.47cm.
                row{4-5}       = {l},
                row{2}         = {2cm},         % Pengaturan tinggi baris kedua = 2cm.
                hlines, vlines                  % add borders to all cells
            }
            Tanda Tangan    & \textbf{Disusun}                  & \textbf{Diperiksa}                       &                              &                              & \\
                            &                                   &                                          &                              &                              & \\
            Tanggal         &                                   &                                          &                              &                              &  \\
            Jabatan         & QA Supervisor                     & Programmer Supervisor                    & Programmer Supervisor        & Programmer Supervisor        & Programmer Supervisor \\
            Nama            & Rafid M. Salahudin                & Mf Sucianto                              & Mansur                       & Azika Syahputra Azwar        & Bima Samudra
        \end{tblr}
    \end{table}

    \vspace{-2em}

    \begin{table}[H]
        \centering
        \small % make the font smaller
        \begin{tblr}{
                cell{1}{2-6}   = {c},
                cell{1}{2}     = {c=5}{},
                cell{1}{1}     = {r=2}{1.75cm}, % Baris 1 kolom 1 = Span 2 baris, lebar 1.5cm.
                cell{4-5}{2-6} = {2.55cm},      % Pengaturan lebar kolom 2-6 pada baris 4-5 = 2.47cm.
                row{4-5}       = {l},
                row{2}         = {2cm},         % Pengaturan tinggi baris kedua = 2cm.
                hlines, vlines                  % add borders to all cells
            }
            Tanda Tangan    & \textbf{Diperiksa}                &                                &                           &                                & \\
                            &                                   &                                &                           &                                & \\
            Tanggal         &                                   &                                &                           &                                &  \\
            Jabatan         & Technical Support Supervisor      & System Integrator Coordinator  & SoftDev assc. Manager     & Infra Assc. \& TS Manager      & Infra Assc. \& TS Manager \\
            Nama            & Burhanudin Zali                   & Binto Widodo                   & Prabu Karana              & Muhammad Nur Alam              & Muhammad Azharuddin
        \end{tblr}
    \end{table}

    \vspace{-2em}

    \begin{table}[H]
        \centering
        \small % make the font smaller
        \begin{tblr}{
                cell{1}{2-5}   = {c},
                cell{1}{2}     = {c=3}{},       % Baris 1 kolom 2 = Span 3 kolom lebar flexibel.
                cell{1}{6}     = {c=2}{},       % Baris 1 kolom 6 = Span 2 kolom lebar flexibel.
                cell{1}{1}     = {r=2}{1.75cm}, % Baris 1 kolom 1 = Span 2 baris, lebar 1.5cm.
                cell{4-5}{2-5} = {3.3cm},       % Pengaturan lebar kolom 2-5 pada baris 4-5 = 3.19cm.
                row{4-5}       = {l},
                row{2}         = {2cm},         % Pengaturan tinggi baris kedua = 2cm.
                hlines, vlines                  % add borders to all cells
            }
            Tanda Tangan    & \textbf{Diperiksa}                &                                  &                                  & \textbf{Disetujui} \\
                            &                                   &                                  &                                  & \\
            Tanggal         &                                   &                                  &                                  & \\
            Jabatan         & Softdev Manager                   & Softdev Manager                  & Assc.Group Infra \& TS Manager   & Group MIS Manager \\
            Nama            & Yandi Prabowo                     & Ahmad Taufiq                     & Richard Stefano Ricci            & Eddy Chandra
        \end{tblr}
    \end{table}

    \hspace{-1em}\textbf{BENTUK PERUBAHAN:}
    \vspace{5em} % Baris ini bisa dihapus saja bila sudah ada konten-nya.
    \texttt{}
    \begin{picture}(0,0)
        \put(-147,-45){\noindent\rule{17.25cm}{0.35pt}} % 17.25cm wide and 0.35pt thick
    \end{picture}

    \hspace{-1em}\textbf{ALASAN PERUBAHAN:}
    \vspace{5em} % Baris ini bisa dihapus saja bila sudah ada konten-nya.
    \texttt{}
    \begin{picture}(0,0)
        \put(-145,-45){\noindent\rule{17.25cm}{0.35pt}} % 17.25cm wide and 0.35pt thick
    \end{picture}

    \hspace{-1em}\textbf{TINDAKAN SEMASA PERALIHAN:}
    \vspace{5em} % Baris ini bisa dihapus saja bila sudah ada konten-nya.

    \hspace{-1em}\textbf{DISTRIBUSI:}
    \begin{picture}(0,0)
        \put(-82,12){\noindent\rule{17.25cm}{0.35pt}} % 17.25cm wide and 0.35pt thick
    \end{picture}
    \begin{picture}(0,0)
        \put(-85,-5){\noindent\rule{17.25cm}{0.35pt}} % 17.25cm wide and 0.35pt thick
    \end{picture}
    \begin{picture}(0,0)
        \put(-89,-55){\noindent\rule{17.25cm}{0.35pt}} % 17.25cm wide and 0.35pt thick
    \end{picture}
    \begin{picture}(0,0)
        \put(162,-55){\tikz \draw[dotted] (0,0) -- (0,1.7);} % Draw a dotted vertical line
    \end{picture}

    \begin{longtable}{c p{1.5cm} p{4.23cm} c p{1.5cm} p{4.23cm}}
        %\hline
        \textbf{SALINAN} & \textbf{BAGIAN} & \textbf{KETERANGAN} & \textbf{SALINAN} & \textbf{BAGIAN} & \textbf{KETERANGAN} \\%\hline
        \endfirsthead
        \multicolumn{6}{c}%
        {{\bfseries Lanjutan dari tabel Distribusi halaman sebelumnya}} \\
        %\hline
        \textbf{SALINAN} & \textbf{BAGIAN} & \textbf{KETERANGAN} & \textbf{SALINAN} & \textbf{BAGIAN} & \textbf{KETERANGAN} \\%\hline
        \endhead
        \hline \multicolumn{6}{|r|}{{Bersambung pada halaman berikut-nya}} \\%\hline
        \endfoot
        %\hline
        \endlastfoot
        \tiny & & & & & \\
        \#11 & MIS & Management Information System & & & \\%\hline
    \end{longtable}

    \newpage

    \center\textbf{SEJARAH PERUBAHAN}

    \renewcommand{\arraystretch}{1.2} % Adjust this value as needed
    \begin{longtable}{|c|c|r p{12cm}|}
        % Bila setelah ada penambahan konten sejarah perubahan, kemungkinan lebar tabel ini akan berubah.
        % Ubah ukuran kolom terakhir agar lebarnya sama dengan lebar heading SOP, misalnya menjadi dari 11.4cm
        % menjadi 10.5cm.
        \hline
        \textbf{TANGGAL} & \textbf{REVISI} & \multicolumn{2}{c|}{\textbf{URAIAN PERUBAHAN}} \\ \hline
        \endfirsthead
        \multicolumn{4}{c}%
        {{\bfseries Lanjutan dari halaman sebelumnya}} \\
        \hline
        \textbf{TANGGAL} & \textbf{REVISI} & \multicolumn{2}{c|}{\textbf{URAIAN PERUBAHAN}} \\ \hline
        \endhead
        \hline \multicolumn{4}{|r|}{{Lanjut ke halaman berikutnya}} \\ \hline
        \endfoot
        \hline
        \endlastfoot
        &&&\\\hline &&&\\\hline &&&\\\hline &&&\\\hline &&&\\\hline &&&\\\hline &&&\\\hline &&&\\\hline &&&\\\hline &&&\\\hline &&&\\\hline &&&\\\hline &&&\\\hline &&&\\\hline &&&\\\hline &&&\\\hline &&&\\\hline &&&\\\hline &&&\\\hline &&&\\\hline &&&\\\hline &&&\\\hline &&&\\\hline &&&\\\hline &&&\\\hline &&&\\\hline &&&\\\hline &&&\\\hline &&&\\\hline &&&\\\hline &&&\\\hline &&&\\\hline &&&\\\hline &&&\\\hline &&&\\\hline &&&\\\hline &&&\\\hline &&&\\\hline &&&\\\hline &&&\\\hline &&&\\\hline &&&\\\hline &&&\\\hline &&&\\\hline &&&\\\hline &&&\\\hline
        % NOTE:
        % Setiap ada menambahkan baris baru, cek kembali pada file pdf yang
        % dihasilkan pada halaman SEJARAH PERUBAHAN untuk memastikan longtable tidak
        % span ke halaman berikutnya bila pada bagian bawah tabel tersebut masih
        % adalah berupa baris-baris kosong. Bila hal tersebut terjadi, maka kurangi
        % jumlah &&&\\\hline di atas.
    \end{longtable}

    \newpage

    \begin{enumerate}
        % Gunakan \\\vspace{1.5ex} pada setiap 1st level item bila di konten di
        % bawah-nya bukan berupa enumerate. Bila adalah berupa enumerate, maka
        % hapus \\\vspace{1.5ex} tersebut.
        \item \textbf{TUJUAN}\\\vspace{1.5ex}

        \item \textbf{RUJUKAN}
            \begin{enumerate}
                \item ...
                \item ...
            \end{enumerate}

        \item \textbf{DEFINISI}
            % Set the row height stretch
            \renewcommand{\arraystretch}{2} % Adjust the value as needed
            \begin{longtable}{p{8pt} r p{3cm} c p{10.98cm}}
                & 4.1. & \textit{...} & : & ...\\
                & 4.2. & \textit{...} & : & ...\\
                & 4.3. & \textit{...} & : & ...\\
            \end{longtable}

        \item \textbf{TANGGUNG JAWAB}
            \begin{enumerate}
                \item ...
                \item ...
            \end{enumerate}

        \item \textbf{PROSEDUR}
            \begin{enumerate}
                \item ...
                \item ...
            \end{enumerate}

        \item \textbf{PENGECUALIAN}
        \item \textbf{REKAMAN}
        \item \textbf{LAMPIRAN}
    \end{enumerate}
\end{document}
