\documentclass[12pt]{npl.doc}
\usepackage{amsmath} % for \text
\usepackage{longtable} % Required for the longtable environment
\usepackage{float}
\usepackage{tabularray}
\usepackage{nameref} % Required package for using \nameref
\usepackage{tcolorbox}
\usepackage{background}
\usepackage{hyperref}
\title{Judul SOP}
\revisiondate{10 Februari 2025}
\revision{0.0}
\soinumber{NPL-SOP-20-IS-000000-0}

\renewcommand{\theLogoValue}{npl.png}
\renewcommand{\figurename}{Gambar} % Change "Figure" to "Gambar"

% ---------------------------------------------------------------------------
% CUSTOM HYPHENATION
% ---------------------------------------------------------------------------
% Silahkan tambahkan custom pemenggalan kata  di dalam \hyphenation di bawah
% ini bila menemukan ada pemenggalan kata pada generated .pdf file yang salah
% di SOP/I ini.
% ---------------------------------------------------------------------------
\hyphenation{
    pe-ngen-da-li
    pe-mang-ka-san
}
% ---------------------------------------------------------------------------

\begin{document}
    \SetBgContents{}
    % Set up the background text
    \backgroundsetup{
      scale=7,              % Scale the text
      color=red,            % Color of the text
      opacity=0.15,         % Opacity of the text
      position={1.2, -1.5}, % Positioning
      contents=DRAFT        % The text to display
    }
    % ========================================
    % Saat sudah siap untuk proses pengesahan:
    % ----------------------------------------
    % 1. Comment, dengan menyisipkan % pada block \backgroundsetup{ s/d } di
    %    atas. Dengan editor VIM, cara-nya sebagai berikut:
    %     a) Saat di mode NORMAL, arahkan kursor ke posisi \ pada
    %        \backgroundsetup{ di atas.
    %     b) Tekan CTRL+v, lalu tekan j beberapa kali hingga kursor sampai pada
    %        posisi } yang sejajar dengan \ di bawah-nya.
    %     c) Shift+i, ketik %, dan lanjutkan dengan tekan Escape.
    % 2. Hapus % block \backgroundsetup{ s/d } di bawah. Dengan editor VIM,
    %    cara-nya sebagai berikut:
    %     a) Saat di mode NORMAL, arahkan kursor ke ke posisi \ pada
    %        \backgroundsetup{ di bawah.
    %     b) Tekan CTRL+v, lalu tekan j beberapa kali hingga kursor sampai pada
    %        posisi } yang sejajar dengan \ di bawah-nya.
    %     c) Tekan d.
    %
    %\backgroundsetup{
    %  scale=5,              % Scale the text
    %  color=red,            % Color of the text
    %  opacity=0.15,         % Opacity of the text
    %  position={1.7, -2}, % Positioning
    %  contents=CONFIDENTIAL % The text to display
    %}

    \begin{table}
        \centering
        \small % make the font smaller
        \begin{tblr}{
                cell{1}{2-6}   = {c},
                cell{1}{3}     = {c=4}{},      % Baris 1 kolom 3 = Span 4 kolom lebar flexibel.
                cell{1}{5}     = {c=2}{},      % Baris 1 kolom 5 = Span 2 kolom lebar flexibel.
                cell{1}{1}     = {r=2}{1.5cm}, % Baris 1 kolom 1 = Span 2 baris, lebar 1.5cm.
                cell{4-5}{2-6} = {2.47cm},     % Pengaturan lebar kolom 2-6 pada baris 4-5 = 2.47cm.
                row{4-5}       = {l},
                row{2}         = {2cm},        % Pengaturan tinggi baris kedua = 2cm.
                hlines, vlines                 % add borders to all cells
            }
            Tanda Tangan    & \textbf{Disusun}                  & \textbf{Diperiksa}                       &                              &                              & \\
                            &                                   &                                          &                              &                              & \\
            Tanggal         &                                   &                                          &                              &                              &  \\
            Jabatan         & Infra Assc. \& TS Manager         & System Integrator Coordinator            & Infra Assc. \& TS Manager    & SoftDev assc. Manager        & Softdev Manager \\
            Nama            & Muhammad Azharuddin               & Binto Widodo                             & Muhammad Nur Alam            & Prabu Karana                 & Yandi Prabowo
        \end{tblr}
    \end{table}

    \begin{table}
        \centering
        \small % make the font smaller
        \begin{tblr}{
                cell{1}{2-6}   = {c},
                cell{1}{2}     = {c=2}{},
                cell{1}{4}     = {c=3}{},
                cell{1}{1}     = {r=2}{1.5cm}, % Baris 1 kolom 1 = Span 2 baris, lebar 1.5cm.
                cell{4-5}{2-6} = {2.47cm},     % Pengaturan lebar kolom 2-6 pada baris 4-5 = 2.47cm.
                row{4-5}       = {l},
                row{2}         = {2cm},        % Pengaturan tinggi baris kedua = 2cm.
                hlines, vlines                 % add borders to all cells
            }
            Tanda Tangan    & \textbf{Diperiksa}                &                                & \textbf{Disetujui}           &                      & \\
                            &                                   &                                &                              &                      & \\
            Tanggal         &                                   &                                &                              &                      &  \\
            Jabatan         & Softdev Manager                   & Assc.Group Infra \& TS Manager & Business Improvement Manager & Group MIS Manager    & Director \\
            Nama            & Ahmad Taufiq                      & Richard Stefano Ricci          & Devina Nur Wulan             & Eddy Chandra         & Roy Lembong
        \end{tblr}
    \end{table}

    \section*{BENTUK PERUBAHAN:}
    \vspace{5em} % Baris ini bisa dihapus saja bila sudah ada konten-nya.

    \section*{ALASAN PERUBAHAN:}
    \vspace{5em} % Baris ini bisa dihapus saja bila sudah ada konten-nya.

    \section*{TINDAKAN SEMASA PERALIHAN:}
    \vspace{5em} % Baris ini bisa dihapus saja bila sudah ada konten-nya.

    \section*{DISTRIBUSI:}

    \begin{longtable}{|c|p{1.5cm}|p{3.6cm}|c|p{1.5cm}|p{3.6cm}|}
        \hline
        \textbf{SALINAN} & \textbf{BAGIAN} & \textbf{KETERANGAN} & \textbf{SALINAN} & \textbf{BAGIAN} & \textbf{KETERANGAN} \\ \hline
        \endfirsthead
        \multicolumn{6}{c}%
        {{\bfseries Lanjutan dari tabel Distribusi halaman sebelumnya}} \\
        \hline
        \textbf{SALINAN} & \textbf{BAGIAN} & \textbf{KETERANGAN} & \textbf{SALINAN} & \textbf{BAGIAN} & \textbf{KETERANGAN} \\ \hline
        \endhead
        \hline \multicolumn{6}{|r|}{{Bersambung pada halaman berikut-nya}} \\ \hline
        \endfoot
        \hline
        \endlastfoot
        \#11 & MIS & Management Information System & & & \\ \hline
    \end{longtable}

    \newpage

    \section*{SEJARAH PERUBAHAN}

    \begin{longtable}{|c|c|r p{11.4cm}|}
        % Bila setelah ada penambahan konten sejarah perubahan, kemungkinan lebar tabel ini akan berubah.
        % Ubah ukuran kolom terakhir agar lebarnya sama dengan lebar heading SOP, misalnya dari 11.4cm
        % menjadi 10.5cm.
        \hline
        \textbf{TANGGAL} & \textbf{REVISI} & \multicolumn{2}{c|}{\textbf{URAIAN PERUBAHAN}} \\ \hline
        \endfirsthead
        \multicolumn{4}{c}%
        {{\bfseries Lanjutan dari halaman sebelumnya}} \\
        \hline
        \textbf{TANGGAL} & \textbf{REVISI} & \multicolumn{2}{c|}{\textbf{URAIAN PERUBAHAN}} \\ \hline
        \endhead
        \hline \multicolumn{4}{|r|}{{Lanjut ke halaman berikutnya}} \\ \hline
        \endfoot
        \hline
        \endlastfoot
        &&&\\\hline &&&\\\hline &&&\\\hline &&&\\\hline &&&\\\hline &&&\\\hline &&&\\\hline &&&\\\hline &&&\\\hline &&&\\\hline &&&\\\hline &&&\\\hline &&&\\\hline &&&\\\hline &&&\\\hline &&&\\\hline &&&\\\hline &&&\\\hline &&&\\\hline &&&\\\hline &&&\\\hline &&&\\\hline &&&\\\hline &&&\\\hline &&&\\\hline &&&\\\hline &&&\\\hline &&&\\\hline &&&\\\hline &&&\\\hline &&&\\\hline &&&\\\hline &&&\\\hline &&&\\\hline &&&\\\hline &&&\\\hline &&&\\\hline &&&\\\hline &&&\\\hline &&&\\\hline &&&\\\hline &&&\\\hline &&&\\\hline &&&\\\hline &&&\\\hline &&&\\\hline &&&\\\hline &&&\\\hline &&&\\\hline
        % NOTE:
        % Setiap ada menambahkan baris baru, cek kembali pada file pdf yang
        % dihasilkan pada halaman SEJARAH PERUBAHAN untuk memastikan longtable tidak
        % span ke halaman berikutnya bila pada bagian bawah tabel tersebut masih
        % adalah berupa baris-baris kosong. Bila hal tersebut terjadi, maka kurangi
        % jumlah &&&\\\hline di atas.
    \end{longtable}

    \newpage

    \section{TUJUAN}

    \section{RUANG LINGKUP}
    \begin{enumerate}
        \item ...
        \item ...
    \end{enumerate}

    \section{RUJUKAN}
    \begin{enumerate}
        \item ...
        \item ...
    \end{enumerate}

    \section{DEFINISI}
    \begin{longtable}{r p{2cm} c p{12.7cm}}
        4.1. & \textit{...} & : & ...\\
        4.2. & \textit{...} & : & ...\\
        4.3. & \textit{...} & : & ...\\
    \end{longtable}

    \section{KETENTUAN UMUM}
    \begin{enumerate}
        \item ...
        \item ...
    \end{enumerate}

    \section{PROSEDUR DAN TANGGUNG JAWAB}
    \begin{enumerate}
        \item ...
        \item ...
    \end{enumerate}

    \section{PENGECUALIAN}
    \section{REKAMAN MUTU}
    \section{LAMPIRAN}
\end{document}
