\documentclass[12pt]{soi_v2}
\usepackage{amsmath} % for \text
\usepackage{longtable} % Required for the longtable environment
\usepackage{float}
\usepackage{tabularray}
\usepackage{nameref} % Required package for using \nameref
\usepackage{tcolorbox}
\usepackage{background}
\title{Judul SOP}
\revisiondate{Fri Nov 01, 2024}
\revision{0}
\soinumber{NPL-SOP-20-IS-XXXXXX-X}

\begin{document}
    \SetBgContents{}
    % Set up the background text
    \backgroundsetup{
      scale=7,              % Scale the text
      color=red,            % Color of the text
      opacity=0.15,         % Opacity of the text
      position={1.2, -1.5}, % Positioning
      contents=DRAFT        % The text to display
    }

    \maketitle

    \begin{table}
        \centering
        \small % make the font smaller
        \begin{tblr}{
          stretch=1.5,
          row{1} = {p},
          cell{1}{1} = {r=2}{1.5cm}, % Pengaturan lebar kolom pertama (Tanda Tangan)
          cell{2}{2} = {4.5cm}, % Pengaturan lebar kolom Disusun
          cell{2}{3} = {5cm},   % Pengaturan lebar kolom Diperiksa
          cell{2}{4} = {4.5cm}, % Pengaturan lebar kolom Disetujui
          hlines, vlines % add borders to all cells
        }
        Tanda Tangan & \textbf{Disusun} & \textbf{Diperiksa} & \textbf{Disetujui} \\
                        & {\\ \\ \\ \\}       &                    & \\
        Tanggal         &                     &                    & \\
        Jabatan         & QA Supervisor & {System Integration Coordinator} & Group MIS Manager  \\
        Nama            & Rafid M. Salahudin & Binto Widodo Pramono              & Eddy Chandra
        \end{tblr}
    \end{table}

    \section*{BENTUK PERUBAHAN:}

    \textit{Belum ada (SOP Baru)}

    \section*{ALASAN PERUBAHAN:}

    \textit{Belum ada (SOP Baru)}

    \section*{TINDAKAN SEMASA PERALIHAN:}

    \textit{Belum ada (SOP Baru)}

    \section*{DISTRIBUSI:}

    \begin{longtable}{|c|p{1.5cm}|p{4cm}|c|p{1.5cm}|p{4cm}|}
        \hline
        \textbf{SALINAN} & \textbf{BAGIAN} & \textbf{KETERANGAN} & \textbf{SALINAN} & \textbf{BAGIAN} & \textbf{KETERANGAN} \\ \hline
        \endfirsthead
        \multicolumn{6}{c}%
        {{\bfseries continued from previous page}} \\
        \hline
        \textbf{SALINAN} & \textbf{BAGIAN} & \textbf{KETERANGAN} & \textbf{SALINAN} & \textbf{BAGIAN} & \textbf{KETERANGAN} \\ \hline
        \endhead
        \hline \multicolumn{6}{|r|}{{Continued on next page}} \\ \hline
        \endfoot
        \hline
        \endlastfoot
        \#1 & SI & System Integration & & & \\ \hline
        \#2 & SD & Software Development & & & \\ \hline
        \#3 & IF & Infrastructure & & & \\ \hline
        \#4 & BOD & Board Of Director & & & \\ \hline
        \#5 & ... & ... & & & \\ \hline
    \end{longtable}

    \newpage

    \section*{SEJARAH PERUBAHAN}

    \begin{longtable}{|c|c|p{11.7cm}|}
        \hline
        \textbf{TANGGAL} & \textbf{REVISI} & \textbf{URAIAN PERUBAHAN} \\ \hline
        \endfirsthead
        \multicolumn{3}{c}%
        {{\bfseries Lanjutan dari halaman sebelumnya}} \\
        \hline
        \textbf{TANGGAL} & \textbf{REVISI} & \textbf{URAIAN PERUBAHAN} \\ \hline
        \endhead
        \hline \multicolumn{3}{|r|}{{Lanjut ke halaman berikutnya}} \\ \hline
        \endfoot
        \hline
        \endlastfoot
                   &   &            \\ \hline
                   &   &            \\ \hline
                   &   &            \\ \hline
                   &   &            \\ \hline
                   &   &            \\ \hline
                   &   &            \\ \hline
                   &   &            \\ \hline
                   &   &            \\ \hline
                   &   &            \\ \hline
    \end{longtable}

    \newpage

    \section{TUJUAN}

    \section{RUANG LINGKUP}
    \begin{enumerate}
        \item ...
        \item ...
    \end{enumerate}

    \section{RUJUKAN}
    \begin{enumerate}
        \item ...
        \item ...
    \end{enumerate}

    \section{DEFINISI}
    \begin{enumerate}
        \item \textbf{bla3x1:} Adalah blablabla satu.
        \item \textbf{bla3x2:} Adalah blablabla dua.
    \end{enumerate}

    \section{KETENTUAN UMUM}
    \begin{enumerate}
        \item ...
        \item ...
    \end{enumerate}

    \section{PROSEDUR \& TANGGUNG JAWAB}
    \begin{enumerate}
        \item ...
        \item ...
    \end{enumerate}

    \section{PENGECUALIAN}
    \section{REKAMAN MUTU}
    \section{LAMPIRAN}
\end{document}
