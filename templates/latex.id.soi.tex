\documentclass[12pt]{sop}
\usepackage{amsmath} % for \text
\usepackage{longtable} % Required for the longtable environment
\usepackage{float}
\usepackage{tabularray}
\usepackage{nameref} % Required package for using \nameref
\usepackage{tcolorbox}
\usepackage{background}
\usepackage{hyperref}
\title{Judul SOI}
\revisiondate{29 November 2024}
\revision{0}
\soinumber{NPL-SOI-MIS-XXX-00-MM.YYYY}

\renewcommand{\theLogoValue}{logo.png}

\begin{document}
    \SetBgContents{}
    % Set up the background text
    \backgroundsetup{
      scale=7,              % Scale the text
      color=red,            % Color of the text
      opacity=0.15,         % Opacity of the text
      position={1.2, -1.5}, % Positioning
      contents=DRAFT        % The text to display
    }
    % ========================================
    % Saat sudah siap untuk proses pengesahan:
    % ----------------------------------------
    % 1. Comment, dengan menyisipkan % pada block \backgroundsetup{ s/d } di
    %    atas. Dengan editor VIM, cara-nya sebagai berikut:
    %     a) Saat di mode NORMAL, arahkan kursor ke posisi \ pada
    %        \backgroundsetup{ di atas.
    %     b) Tekan CTRL+v, lalu tekan j beberapa kali hingga kursor sampai pada
    %        posisi } yang sejajar dengan \ di bawah-nya.
    %     c) Shift+i, ketik %, dan lanjutkan dengan tekan Escape.
    % 2. Hapus % block \backgroundsetup{ s/d } di bawah. Dengan editor VIM,
    %    cara-nya sebagai berikut:
    %     a) Saat di mode NORMAL, arahkan kursor ke ke posisi \ pada
    %        \backgroundsetup{ di bawah.
    %     b) Tekan CTRL+v, lalu tekan j beberapa kali hingga kursor sampai pada
    %        posisi } yang sejajar dengan \ di bawah-nya.
    %     c) Tekan d.
    %
    %\backgroundsetup{
    %  scale=5,              % Scale the text
    %  color=red,            % Color of the text
    %  opacity=0.15,         % Opacity of the text
    %  position={1.7, -2}, % Positioning
    %  contents=CONFIDENTIAL % The text to display
    %}

    \begin{table}
        \centering
        \small % make the font smaller
        \begin{tblr}{
          cell{1}{2-6}   = {c},
          cell{1}{3}     = {c=2}{},      % Baris 1 kolom 3 = Span 2 kolom lebar flexibel.
          cell{1}{5}     = {c=2}{},      % Baris 1 kolom 5 = Span 2 kolom lebar flexibel.
          cell{1}{1}     = {r=2}{1.5cm}, % Baris 1 kolom 1 = Span 2 baris, lebar 1.5cm.
          cell{4-5}{2-6} = {2.47cm},     % Pengaturan lebar kolom 2-4 pada baris 4-5.
          row{4-5}       = {l},
          row{2}         = {2cm},        % Pengaturan tinggi baris kedua
          hlines, vlines                 % add borders to all cells
        }
        Tanda Tangan    & \textbf{Disusun}                  & \textbf{Diperiksa}                       &                              & \textbf{Disetujui}           & \\
                        &                                   &                                          &                              &                              & \\
        Tanggal         &                                   &                                          &                              &                              & \\
        Jabatan         &                                   &                                          & QA Supervisor                & Business Improvement Manager & Group MIS Manager \\
        Nama            &                                   &                                          & Rafid M. Salahudin           & Devina Nur Wulan             & Eddy Chandra
        \end{tblr}
    \end{table}

    \section*{BENTUK PERUBAHAN:}

    \textit{Belum ada (SOI Baru)}

    \section*{ALASAN PERUBAHAN:}

    \textit{Belum ada (SOI Baru)}

    \section*{TINDAKAN SEMASA PERALIHAN:}

    \textit{Belum ada (SOI Baru)}

    \section*{DISTRIBUSI:}

    \begin{longtable}{|c|p{1.5cm}|p{3.6cm}|c|p{1.5cm}|p{3.6cm}|}
        \hline
        \textbf{SALINAN} & \textbf{BAGIAN} & \textbf{KETERANGAN} & \textbf{SALINAN} & \textbf{BAGIAN} & \textbf{KETERANGAN} \\ \hline
        \endfirsthead
        \multicolumn{6}{c}%
        {{\bfseries Lanjutan dari tabel Distribusi halaman sebelumnya}} \\
        \hline
        \textbf{SALINAN} & \textbf{BAGIAN} & \textbf{KETERANGAN} & \textbf{SALINAN} & \textbf{BAGIAN} & \textbf{KETERANGAN} \\ \hline
        \endhead
        \hline \multicolumn{6}{|r|}{{Bersambung pada halaman berikut-nya}} \\ \hline
        \endfoot
        \hline
        \endlastfoot
        \#11 & MIS & Management Information System & & & \\ \hline
    \end{longtable}

    \newpage

    \section*{SEJARAH PERUBAHAN}

    \begin{longtable}{|c|c|p{11.7cm}|}
        \hline
        \textbf{TANGGAL} & \textbf{REVISI} & \textbf{URAIAN PERUBAHAN} \\ \hline
        \endfirsthead
        \multicolumn{3}{c}%
        {{\bfseries Lanjutan dari halaman sebelumnya}} \\
        \hline
        \textbf{TANGGAL} & \textbf{REVISI} & \textbf{URAIAN PERUBAHAN} \\ \hline
        \endhead
        \hline \multicolumn{3}{|r|}{{Lanjut ke halaman berikutnya}} \\ \hline
        \endfoot
        \hline
        \endlastfoot
                   &   &            \\ \hline
                   &   &            \\ \hline
                   &   &            \\ \hline
                   &   &            \\ \hline
                   &   &            \\ \hline
                   &   &            \\ \hline
                   &   &            \\ \hline
                   &   &            \\ \hline
                   &   &            \\ \hline
    \end{longtable}

    \newpage

    \section{TUJUAN}

    \section{RUANG LINGKUP}
    \begin{enumerate}
        \item ...
        \item ...
    \end{enumerate}

    \section{RUJUKAN}
    \begin{enumerate}
        \item ...
        \item ...
    \end{enumerate}

    \section{DEFINISI}
    \begin{enumerate}
        \item \textbf{...:} ...
        \item \textbf{...:} ...
    \end{enumerate}

    \section{DEFINISI SINGKATAN}
    \begin{enumerate}
        \item \textbf{...:} ...
        \item \textbf{...:} ...
    \end{enumerate}

    \section{KETENTUAN UMUM}
    \begin{enumerate}
        \item ...
        \item ...
    \end{enumerate}

    \section{PERSIAPAN}
    \begin{enumerate}
        \item ...
        \item ...
    \end{enumerate}

    \section{PROSEDUR KERJA}
    \begin{enumerate}
        \item ...
        \item ...
    \end{enumerate}

    \section{PENGECUALIAN}
    \section{FLOWCHART}
    \section{LAMPIRAN}
\end{document}
